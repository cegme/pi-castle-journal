\begin{abstract}
The amount of text data has been growing exponentially in recent
years.
State-of-the-art statistical text extraction methods over this data are
likely to contain errors.
Recent work has shown probabilistic databases
can store and query uncertainty over extraction results, however, these systems
do not nately result in a reduction of error.
In this paper we propose \sysName, a system that uses a probabilistic
database as an anchor to execute, optimize and integrate machine and human
computing.  Uncertain fields are crowdsourced with the goal of reducing
uncertainty and improving accuracy.  We use information theory to optimize the
set of questions and a novel Bayesian probabilistic model to integrate
uncertain crowd answers back into the database.  Experiments show promising
results in significantly reducing machine error using very small amounts of
data.  Additionally, probabilistic integration is shown to more effectively
resolve conflicting crowd answers and provide users with the flexibility to
tune the desired trade-off between accuracy and recall according to the need of
applications.
Using crowds to assist machine-learned models proves to be a cost-effective way to
close the ``last mile'' in terms of accuracy for text labeling and
extraction tasks.

\eat{The need to automatically process and store large amounts of
uncertain and imprecise machine learned data has necessitated the
use of Probabilistic Databases (PDBs) which maintain and allow
queries that carry a degree of uncertainty. Another emerging trend
is crowdsourcing incomplete data through frameworks such as Amazon
Mechanical Turk.  In this paper, we introduce \sysName, an uncertain
data management system for sequential labeling of automatically
extracted text data. \sysName is a complete system that blends
uncertainty management with crowdsourcing techniques designed to
clean the data and reduce overall uncertainty.  It represents a
merging of machine and human computation for efficient task
solutions balancing speed, cost, and accuracy.  Work presented here
is applied on automatic bibliographic citation labeling, though the
techniques laid out may be applied in numerous other information
extraction domains.\dzw{need to rewrite}}
\end{abstract}
