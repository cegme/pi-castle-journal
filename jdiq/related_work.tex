\section{Related Work}
\label{sec:related}
There have been a number of previous attempts at combining humans and machines into a unified data management system, but to our knowledge none that have done it probabilistically.  CrowdDB~\cite{DBLP:conf/sigmod/FranklinKKRX11} utilizes humans to process queries that are either missing from the database or computationally prohibitive to calculate.  The crowd is invoked only if an incoming query contains one of the incomplete values.  In contrast, \sysName operates on batches over a complete, but uncertain database, improving accuracy well in advance of queries.  Qurk~\cite{DBLP:conf/sigmod/MarcusWKMM11} crowdsources workflow operations of the database, such as filtering, aggregating, sorting, and joining, but makes no attempt to optimize the workload between both humans and machines.  AskIt!~\cite{DBLP:conf/icde/BoimGMNPT12} provides a similar goal of "selecting questions" to minimize uncertainty given some budget, however their approach is purely for quality control.  CrowdER~\cite{DBLP:journals/pvldb/WangKFF12} uses a hybrid human-machine system to perform entity resolution.  The machine does an initial course pass over the data and the most likely matching pairs are verified through crowdsourcing.  The primary approach of that work is HIT interface optimization as opposed to the specific question selection and answer integration of ours.  DBLife~\cite{DBLP:conf/cidr/DeRoseSCLBDR07} invites mass collaboration to improve a database seeded by machine learning, but selection is done by humans as needed without any means of automatically identifying likely errors.  There is also no means of doing quality control or redundancy checking if humans introduce erroroneous edits.
